\documentstyle[12pt]{article}
\pagestyle{empty}
\topmargin -0.60in
\oddsidemargin 0.0625in
\textheight 9.00in
\textwidth 6.50in
\renewcommand{\baselinestretch}{1.4}
\parskip 0.20in

\begin{document}
\pagestyle{plain}

\begin{center}
\Huge
Adding Diagnostics to XOOPIC
\\[10mm]
\large
Last change: 2016-02-26
\end{center}

\section{A top level explanation of how XOOPIC runs}

The XOOPIC GUI, XGrafix, controls the simulation once it is running.
Based on user input, it displays diagnostics or runs the simulation.

XGrafix causes the simulation to advance a timestep by calling the
XGMainLoop() function, (in {\tt xg/xgmain.cpp}) which in turn calls Physics(),
which executes the numerical model of the simulation, and 
Maintain\_Diagnostics(), which extracts information from the physics
model for examination.

Summarizing:
The user controls XGrafix, XGrafix calls/controls XGMainLoop, which
calls Physics() and Maintain\_Diagnostics().

It is in Maintain\_Diagnostics that one would process raw Physics() data,
such as taking FFT's of electric fields on the mesh, or maintaining
a history of some quantity.  Maintain\_Diagnostics in turn calls history(),
which maintains quantities as a function of time.

\section{Files to modify to add a diagnostic}

These are the files which are relevant to initializing and maintaining
a diagnostic:
\begin{itemize}
  \item {\bf {\tt otools/diagn.cpp}}:  This file contains the function
  Diagnostics::Diagnostics(), which gets initializes the diagnostic arrays and
  sets up histories, and gets 

  \item {\bf {\tt otools/initwin.cpp}}:  This file contains the calls to
  XGrafix2.0, which set up the graphics windows for the diagnostics. 

  \item {\bf {\tt otools/diagn.h}}:  declarations and definitions for the
  diagnostics. It is also a means of giving {\tt initwin.cpp} access to
  variables declared in {\tt diagn.cpp}.
\end{itemize}

For some diagnostics, you might have to modify the files in the {\tt physics/}
to have them provide additional data for your diagnostics.

\section{Example of adding a simple diagnostic}

Suppose a time history of the electrostatic
potential at a point A is needed.  This is a list of the things that would
need to be done to add the diagnostic:

\subsection{Step 1}
in {\tt otools/diagn.cpp}:

At the top of the file, declare a place to store the time history:

{\bf Scalar *phi\_history\_at\_A;}

\subsection{Step 2}
In the function Set\_Diagnostics in {\tt otools/diagn.cpp}, allocate
space for this diagnostic and write zeros to the newly allocated memory.

{\bf  phi\_history\_at\_A = new Scalar[HISTMAX+1];  //allocate memory \\
	 //initialize it to zero: \\
  memset(phi\_history\_at\_A,0,(HISTMAX+1)*sizeof(Scalar)); }

It is important to initialize memory to zero:  XGrafix may operate on the
data before it is valid, (such as trying to display the history before the
simulation has advanced a timestep), and many architectures will crash if
you try to access invalid floating-point data.

\subsection{Step 3}
In the history() function, add the line:

{\bf  phi\_history\_at\_A[hist\_hi] = phi[jm/2][km/2];}

Add this line near the line: 

{\bf  energy\_e[hist\_hi] = Uetot;}
  
in the history() function.  A history of phi at the grid location (jm/2,km/2)
will now be maintained in the array ``phi\_history\_at\_A''.  "jm" and "km"
are the maximum values of j and k respectively, so this diagnostic will
take the potential history near the middle of the device.

\subsection{Step 4}
Also in the history function, add the line
{\bf  phi\_history\_at\_A[i] = phi\_history\_at\_A[k]; }

near the line:

{\bf  energy\_e[i] = energy\_e[k];}

  Since the time history is taken through the entire simulation, but the phi\_history
array has only HISTMAX+1 entries, we must "comb" the time history periodically.
This line ensures that this diagnostic will be properly combed.


\subsection{Step 5}

Add this line to {\tt diagn.h}:

{\bf  extern Scalar *phi\_history\_at\_A;}

\subsection{Step 6}
 Make the call to XGrafix in {\tt initwin.cpp} which will set up a window
to display this data in.

In {\tt initwin.cpp}, add these lines to the InitWin() function:

{\bf  XGSet2D("linlin","phi", "Potential history at point A", "closed", 700,400,
	1.0,1.0,True,True,0,0,0.0,0.0);\\
  XGCurve(t\_array, phi\_history\_at\_A, \&hist\_hi, 1); }

These function calls are from the XGrafix 2.0 library.  Refer to the manual for
XGrafix 2.0 for a description of XGrafix and these function calls.

\subsection{Step 7}

  Recompile xoopic and examine the diagnostic to see if you have carried
out the previous steps correctly.  Adding this diagnostic took 9 lines of code.


\end{document}

